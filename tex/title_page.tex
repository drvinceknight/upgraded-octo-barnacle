\documentclass[numbers,webpdf,imaman]{ima-authoring-template}%

\usepackage{latexsym}
\usepackage{subcaption}
\usepackage{graphicx}
\usepackage{mathptmx}
\usepackage{placeins}
% \usepackage{fullpage}
\usepackage[left=16mm, right=16mm, top=23mm, bottom=23mm]{geometry}
%
%****************************************************************************
% AUTHOR: You may want to use some of these packages. (Optional)
\usepackage{amsmath}
\usepackage{amsfonts}
\usepackage{amssymb}
\usepackage{amsbsy}
\usepackage{amsthm}
\usepackage{booktabs}
\usepackage{epsfig}

\usepackage{caption}
\usepackage{standalone}
\usepackage{tikz}
\usetikzlibrary{arrows}
\usetikzlibrary{decorations.markings}
\usetikzlibrary{calc}
\usetikzlibrary{shapes}

% \usepackage[round]{natbib}
% \usepackage[linesnumbered,lined,ruled,commentsnumbered]{algorithm2e}

\newcommand{\specialcellr}[2][r]{%
  \begin{tabular}[#1]{@{}r@{}}#2\end{tabular}}

%****************************************************************************


%
%****************************************************************************
% AUTHOR: If you do not wish to use hyperlinks, then just comment
% out the hyperref usepackage commands below.

%% This version of the command is used if you use pdflatex. In this case you
%% cannot use ps or eps files for graphics, but pdf, jpeg, png etc are fine.

% \usepackage[
%   colorlinks=true,
%   urlcolor=blue,
%   citecolor=black,
%   anchorcolor=black,
%   linkcolor=black,
% ]{hyperref}
% \usepackage{xurl}

%% The next versions of the hyperref command are used if you adopt the
%% outdated latex-dvips-ps2pdf route in generating your pdf file. In
%% this case you can use ps or eps files for graphics, but not pdf, jpeg, png etc.
%% However, the final pdf file should embed all fonts required which means that you have to use file
%% formats which can embed fonts. Please note that the final PDF file will not be generated on your computer!


%%\usepackage[dvips,colorlinks=true,urlcolor=blue,citecolor=black,%
%% anchorcolor=black,linkcolor=black]{hyperref}
%****************************************************************************


%
%****************************************************************************
%*
%* AUTHOR: YOUR CALL!  Document-specific macros can come here.
%*
\setcitestyle{authoryear,open={(},close={)}} %Citation-related commands
%****************************************************************************
% Packages added manually
\usepackage{xcolor}
\usepackage{multirow}
%#########################################################
%*
%*  The Document.
%*
\begin{document}

\DOI{DOI HERE}
\copyrightyear{2023}
\vol{00}
\pubyear{2024}
\access{Advance Access Publication Date: Day Month Year}
\appnotes{Paper}
\copyrightstatement{Published by Oxford University Press on behalf of the Institute of Mathematics and its Applications. All rights reserved.}
\firstpage{1}

\title[Optimising Heterogeneous Ambulance Fleet Allocations in Jakarta]{Optimising Heterogeneous Ambulance Fleet Allocations in Jakarta}


\author{
Geraint Ian Palmer*, Mark Tuson, Vincent Knight, Paul Harper, and Sarie Brice
\address{
    \orgdiv{School of Mathematics},
    \orgname{Cardiff University},
    \orgaddress{\street{Abacws},
    \postcode{CF24 4AG},
    \state{Cardiff, Wales},
    \country{United Kingdom}
}}}
\author{Leanne Smith
\address{
    \orgname{Welsh Ambulance Services NHS Trust},
    \orgaddress{\street{Beacon House},
    \postcode{NP44 3AB},
    \state{Cwmbran, Wales},
    \country{United Kingdom}
}}}
\author{Daniel Gartner
\address{
    \orgname{Aneurin Bevan University Health Board},
    \orgaddress{\street{Lodge Road},
    \postcode{NP18 3XQ},
    \state{Caerleon, Wales},
    \country{United Kingdom}
}}}


\corresp[*]{Corresponding author: \href{email:palmergi1@cardiff.ac.uk}{palmergi1@cardiff.ac.uk}}

\received{Date}{0}{Year}
\revised{Date}{0}{Year}
\accepted{Date}{0}{Year}

\abstract{Ambulance services have a duty of care to the clinical outcomes of the
population they serve, and therefore aim to maximise the chances of survival and
improve patient outcomes following a medical emergency. Despite this, although
the ambulance allocation problem has been widely studied, it has predominantly
focused on minimising response times or maximising coverage alone, and not
explicitly for considering patient outcomes. In this paper we propose a
modelling approach to consider where to best allocate different types of
emergency response vehicles in order to maximise patient outcomes within a
heterogeneous population. To achieve this, we develop a metaheuristic algorithm
for finding better fleet allocations which is used in conjunction with a
discrete-event simulation model of ambulance services with heterogeneous
vehicles. A major contribution of this metaheuristic is the numerical solution
of a system of equations to approximate the utilisation of vehicles.
Traditionally this utilisation is problematic as it is both an input and an
output of the allocation of vehicles. Our approach is informed by, and tested
on, real-world data from Jakarta, Indonesia. Using our developed models,
decision makers are better able to understand ambulance fleet capacity needs and
allocations, and their impact on patient outcomes.}

\keywords{Ambulance Planning; Emergency Medical Services; Simulation;
Health Care; Optimisation; Mathematical Modelling}


\maketitle

\end{document}