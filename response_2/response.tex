\documentclass{article}

\usepackage{fullpage}
\usepackage{parskip}
\usepackage{setspace}
\usepackage{amsmath}
\usepackage{amsfonts}
\usepackage{amsthm}
\usepackage{hyperref}

\title{Responses to Reviewers}
\author{}
\date{}


\begin{document}

\maketitle

We would like to express our gratitude to the editor, associate editor, and reviewers for facilitating the review process, and for their useful suggestions and remarks on the manuscript.

This document will address the specific concerns raised.

\section{Responses to the Editor}


There were no comments from the Editor.



\section{Responses to the Associate Editor}
\begin{quote}
``Thank you for your revised manuscript. While one reviewer is satisfied with
    the changes, the second reviewer has raised additional concerns that require
    further clarification. It is clear that you have addressed many of the
    previous remarks; however, some points still need more detailed responses.
    Please provide a clear and thorough explanation for the points raised by the
    second reviewer in your next revision. We look forward to your updated
    manuscript.''
\end{quote}

We are delighted to see that we have addressed all the points raised by reviewer
2. Through this response we will endeavour to demonstrate clearly and precisely
how we have addressed the remaining points raised by reviewer 1.

\section{Responses to the Reviewers}

\subsection{Comments from Reviewer 1}

\begin{itemize}

\item Reviewer 1's comment from the first review:

\begin{quote}
``No measures of variance are given for the simulation results in Tables 1
and 4. Especially since repeated simulation results were conducted, measu-
res of variance should be included to investigate the statistical significance
of the results.''
\end{quote}

\item Our response to reviewer 1's comment from the first review:

Thank you for the comment, standard deviations of mean response times are now
given in Table 1 and 4. This is a particularly useful inclusion as the
large variance in the response times explains the discrepencies in the
simulated and expected survivals.

\item Review 1's new comment:

\begin{quote}
Not addressed: The standard deviation of the response time was inclu-
ded, but no information about the variance of the results over the eight
replications was given. To obtain knowledge about the reliability of the
simulation results, some information should be given about how much the
simulation results differed between the eight replications mentioned on
page 17.
\end{quote}

A short discussion about the variability over the repetitions (including
variance) has now been included. 
% TODO VK: We need to do this, I believe it is only Geraint who has the data.

\item Reviewer 1's comment from the first review:

\begin{quote}
``Unclear, why the subchapter 4.1 on survival functions is necessary, sin-
ce traditional coverage targets are used in the case study and the model
formulation does not depend on characteristics of the survival functions.
While survival functions are an important performance metric in the re-
viewers’ opinion, it is not clear why the reader needs this information to
follow the remainder of the pape.''
\end{quote}

\item Our response to reviewer 1's comment from the first review:

We disagree with the reviewer here. In the case study, traditional targets such
a mean response times are reported alongside survival measures, based on
the survival functions. However, a key part of the paper is the
optimisation process, and all scenarios compared are `optimised'
scenarios that result from this process. The objective function of the
optimisation process maximises survival, based on survival functions.
This is evnidenced in Equations 6 and 7, which are component parts of
the objective function given in Equation 5. Therefore the survival
functions are central to the work presented.

\item Review 1's new comment:

\begin{quote}
''Not addressed: It was expanded upon why survival functions can be sui-
table performance metrics. In the case study however, coverage targets
were used (8 min target for A1 patients, 15 min target for A2 patients
and 60 min target for B patients). It is therefore still unclear, why the
reader needs the information in chapter 4.1. Additionally, there is no dis-
cussion as to why coverage targets were chosen in the case study, after
it was argued, that survival functions can solve some drawbacks of using
coverage targets.``
\end{quote}

and

\begin{quote}
    ``
There is contradicting information about which survival functions were
used in the case study. On page 15 and 16, response time targets for the
different categories are listed. This implies to the reader, that step-wise
survival functions, in other words coverage targets, were used in the case
study. On page 19, line 44 it is stated, that Equation 3 was used as the
survival function for A1 patients. It needs to be clarified, which survival
functions were used in the MESLMHPHF in the case study both for the
optimization metaheuristic and the evaluation of the results.
    ''
\end{quote}

\item Our response:

We have added a sentence to the text to clarify that despite the policy target
of 8 minutes for the A1 patients, we have in fact used the survival function of
equation (3) for A1 patients. This was not explicit and we appreciate the
reviewer insisting this point.

\item Reviewer 1's comment:

\begin{quote}
``The simulated survival probabilities in Table 4 show, that the survival for
A1 patients was not improved in any of the four scenarios in the simu-
lation. As the simulation model is a more realistic model of the actual
1
EMS system in comparison to the MESLMHPHF, the simulated survi-
val probabilities carry more weight in interpreting the results compared
to the expected survival probabilities calculated with the MESLMHPHF.
On page 19, line 42 - 44 however it is stated, that ‘in particular A1 pati-
ents are benefitting from the improved allocations’. It is unclear, how the
authors draw this conclusion, when the simulation results show worse or
equal survival probabilities for A1 patients in all scenarios.''
\end{quote}

The reviewer raises a good point here. We have modified the text to hopefully
clarify that:

\begin{itemize}
    \item The sentence regard A1 patients meant to reflect more on the
        efficiency and accuracy of the optimisation heuristic which we find is
        one of the main contributions of this paper (specifically the novel
        approach used to deal with the circular nature of utilisation and
        allocation).
    \item The optimised allocation give more robust allocations under increasing
        demand although we admit that there is not enough numerical evidence to
        be sure of this effect.
\end{itemize}

Furthermore we have more explicitly referred to our explanation about the cause
of this discrepancy.

Fundamentally, again, the reviewer raises a valid and interesting avenue for investigation and
one that we feel
could not be precisely considered without the work in this paper. We hope to
tackle building a more accurate objective function, that better captures the
stochastic effects, that could be then optimised with our novel algorithm. This,
does not however fit in this paper. If the reviewer would like us to make that
more explicit and clear in the text we would be happy to follow any guidance
they feel is necessary.

\item Reviewer 1's comment:

    \begin{quote}
        ``      Equations 8 and 9: It is stated, that \(\beta_{pa_1a_2}\) indicates, that a vehicle from
        \(a_1\) can reach \(p\) quicker, than a vehicle from \(a_2\). The equations 8 and 9
        however use \(\leq\) and not \(<\) , which is not consistent with the description.
Has been fixed for \(\beta_{pa_1a_2}\) in Equation 8, but not for \(R_{pa_1a_2}\) in Equation
9. ''
    \end{quote}
    This has been fixed.
    % TODO Check with Geraint that this is correct.

\end{itemize}

\subsection{Comments from Reviewer 2}

There were no further comments from Reviewer 2.

\end{document}
